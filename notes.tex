\documentclass[a4paper,twocolumn,10pt]{article}
\usepackage[utf8]{inputenc}

% I often type this on OSX, so ignore it.
\DeclareUnicodeCharacter{00A0}{ }

\usepackage[english]{babel}
\usepackage{times}
\usepackage[T1]{fontenc}
\usepackage{parskip}
\usepackage{amsmath}
\usepackage{amssymb}
\usepackage{varioref}
\usepackage[hidelinks]{hyperref}

% Double brackets
\usepackage{stmaryrd}

% Symbols for footnotes
\usepackage[symbol*]{footmisc}

\title{Notes for M.A.~Armstrong's \textit{Groups and symmetry}}
\author{Christian Stigen}
\date{March 2016}

\begin{document}
  \maketitle
  \section{Symmetries of the Tetrahedron}
  \paragraph{Symmetry group} Captures the rules of how symmetries combine for a
  given object.

  \paragraph{Order of operations} In the \textit{product}\footnote{Rotations,
  flips, multiplications, additions, etc. Same order as functional
  composition.} $xyz$, do $z$ first, then $y$ and finally $x$. If order doesn't
  matter in $G$, it's commutative (or \textit{abelian}). Remember to label
  geometric vertices.

  \section{Axioms}
  \paragraph{Group}  Set $G$ with \textit{multiplication} (addition,
  rotation, etc.) satisfying
  \begin{itemize}
    \item \textit{associativity}, i.e.~$(xy)z = x(yz)$
    \item \textit{identity element} $e \in G$ such that $xe=x=ex$
    \item \textit{inverse} $e \in G$ such that $x^{-1}x=e=xx^{-1}$
  \end{itemize}

  \paragraph{Properties common to all groups}
  \begin{itemize}
    \item The identify element of a group is unique.
    \item The inverse of each element of a group is unique.
  \end{itemize}

  \section{Numbers}
  \paragraph{Addition of $\mathbb{Z}, \mathbb{Q}, \mathbb{R}, \mathbb{C}$}
  \begin{itemize}
    \item Identity is zero
    \item $-x$ is the inverse
  \end{itemize}
  \paragraph{Multiplication}
  \begin{itemize}
    \item For $\mathbb{Q}-\{0\}, \mathbb{Q}^+, \mathbb{R}-\{0\}, \mathbb{R}^+,
      \{+1,-1\}, \mathbb{C}-\{0\}, \mathcal{C}\footnote{Complex numbers of
      modulus 1}, \{\pm 1, \pm i\}$: $\underline{e=1 \,\text{and}\, x^{-1}=1/x}$.
  \end{itemize}
  \paragraph{$\mathbb{Z}$ under addition modulus $n$} $e=0, x^{-1}=n-x \,\text{for}\,
  x\ne0$, finite \textit{abelian} group and denoted $\mathbb{Z}_n$.
  \paragraph{$\mathbb{Z}$ under multiplication modulus $n$} Requires $n$ to be
  prime.

\end{document}

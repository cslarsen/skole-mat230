\documentclass[a4paper,12pt]{article}
\usepackage[utf8]{inputenc}

% I often type this on OSX, so ignore it.
\DeclareUnicodeCharacter{00A0}{ }

\usepackage[english]{babel}
\usepackage{times}
\usepackage[T1]{fontenc}
\usepackage{parskip}
\usepackage{amsmath}
\usepackage{amssymb}
\usepackage{varioref}
\usepackage[hidelinks]{hyperref}

% Double brackets
\usepackage{stmaryrd}

% Symbols for footnotes
\usepackage[symbol*]{footmisc}

\title{Notes for M.A.~Armstrong,\\\Huge\textit{Groups and symmetry}}
\author{Christian Stigen}
\date{March 2016}

\begin{document}
  \maketitle
  \section*{Chapter 1}
  \paragraph{Order of operations} In $xyz$, do $z$ first, then $y$ and finally
  $x$. $xy$ means taking the <<product>> (e.g., addition, rotation, etc.) of
  $x$ and $y$. For example, if $r$ is to rotate a cube $90^\circ$
  counter-clockwise, and $s$ is to flip it $180^\circ$ along the axis going
  through two of its oppositely faced sides, then the product $sr$ means rotate
  first, then flip.

  \section*{Chapter 2}
  \paragraph{Group}  Set $G$ with <<multiplication>> (e.g., addition, rotation,
  etc.) satisfying
  \begin{itemize}
    \item \textit{associativity}: $(xy)z = x(yz)$
    \item \textit{identity element} $e \in G$: $xe=x=ex$
    \item \textit{inverse} $e \in G$: $x^{-1}x=e=xx^{-1}$
  \end{itemize}



    \paragraph{Properties common to all groups}
    \begin{itemize}
      \item The identify element of a group is unique.
      \item The inverse of each element of a group is unique.
    \end{itemize}

\end{document}

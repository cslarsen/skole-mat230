\documentclass[a4paper,twocolumn,10pt]{article}
\usepackage[utf8]{inputenc}

% I often type this on OSX, so ignore it.
\DeclareUnicodeCharacter{00A0}{ }

\usepackage[english]{babel}
\usepackage{times}
\usepackage[T1]{fontenc}
\usepackage{parskip}
\usepackage{amsmath}
\usepackage{amssymb}
\usepackage{varioref}
\usepackage[hidelinks]{hyperref}

% Double brackets
\usepackage{stmaryrd}

% Symbols for footnotes
\usepackage[symbol*]{footmisc}

\title{Notes for M.A.~Armstrong's \textit{Groups and Symmetry}}
\author{Christian Stigen}
\date{March 2016}

\begin{document}
  \maketitle
  \section{Symmetries of the Tetrahedron}
  \paragraph{Symmetry group} Captures the rules of how symmetries combine for a
  given object.

  \paragraph{Order of operations} In the \textit{product}\footnote{Rotations,
  flips, multiplications, additions, etc. Same order as functional
  composition.} $xyz$, do $z$ first, then $y$ and finally $x$. If order doesn't
  matter in $G$, it's commutative (or \textit{abelian}). Remember to label
  geometric vertices.

  \section{Axioms}
  \paragraph{Group}  Set $G$ with \textit{multiplication} (addition,
  rotation, etc.) satisfying
  \begin{itemize}
    \item \textit{associativity}, i.e.~$(xy)z = x(yz)$
    \item \textit{identity element} $e \in G$ such that $xe=x=ex$
    \item \textit{inverse} $e \in G$ such that $x^{-1}x=e=xx^{-1}$
  \end{itemize}

  \paragraph{Properties common to all groups}
  \begin{itemize}
    \item The identify element of a group is unique.
    \item The inverse of each element of a group is unique.
  \end{itemize}

  \section{Numbers}
  \paragraph{Addition of $\mathbb{Z}, \mathbb{Q}, \mathbb{R}, \mathbb{C}$}
  \begin{itemize}
    \item Identity is zero
    \item $-x$ is the inverse
  \end{itemize}
  \paragraph{Multiplication}
  \begin{itemize}
    \item For $\mathbb{Q}-\{0\}, \mathbb{Q}^+, \mathbb{R}-\{0\}, \mathbb{R}^+,
      \{+1,-1\}, \mathbb{C}-\{0\}, \mathcal{C}\footnote{Complex numbers of
      modulus 1}, \{\pm 1, \pm i\}$: $\underline{e=1 \,\text{and}\, x^{-1}=1/x}$.
  \end{itemize}
  \paragraph{$\mathbb{Z}$ under addition modulus $n$} $e=0, x^{-1}=n-x \,\text{for}\,
  x\ne0$, finite \textit{abelian} group and denoted $\mathbb{Z}_n$.
  \paragraph{$\mathbb{Z}$ under multiplication modulus $n$} Requires $n$ to be
  prime.

  \section{Dihedral Groups}
  When $n\geq3$ we can manufacture a plate whish has $n$ equal sides. These are
  the non-commutative \textit{dihedral rotational symmetry groups} $D_n$. E.g.
  $D_3 = \{e,r,r^2,s,rs,r^2s\}$. $x^mx^n=x^{m+n}$ and $(x^m)^n=x^{mn}$ provided
  we interpret $x^0=e$. For any multiplication table, each element in $G$
  appears only once in every given column or row.

  $r^n=e$, $s^2=e$, $sr=r^{n-1}s$, $r^{n-1}=r^{-1}$, etc.

  Each element is of form $r^a$, $r^as$ where $0\leq a\leq n-1$.

  For $k=a+_nb$, $r^ar^b=r^k$ and $r^a(r^bs)=r^ks$.  For $l=a+_n(n-b)$,
  $(r^as)r^b=r^ls$ and $(r^as)(r^bs)=r^l$ --- thus $r$ and $s$
  \textit{generate} $D_n$.

  The \textit{order} $|G|$ is the number of elements in the group. If $x^n=e$,
  then $x$ has \textit{finite} order $n$ when $n$ is the smallest such $n$.

  \section{Subgroups and Generators}
  A \textit{subgroup} of $G$ is a subset of $G$ which itself forms a group
  under the multiplication of $G$. For $H$ to be a subgroup of $G$, $H<G$:
  \begin{itemize}
    \item $xy \in G$ for any $x,y \in H$
    \item $e_H \in G$
    \item For any $x \in H$, $x^{-1} \in G$
    \item No need to check associative law.
  \end{itemize}

  \paragraph{Subgroup generated by $x$, or $\langle x \rangle$}
  For an element $x$ in $G$, the set of all $x^n$ is a subgroup of $G$
  (remember $x^0=e$). Finite order $m$ means $x^0=e, x^1, \ldots, x^{m-1}$.
  So order of $x\in G$ is precisely the order of $\langle x\rangle$. If
  $\langle x\rangle=G$, i.e., generates all of $G$, then $G$ is a
  \textbf{\textit{cyclic group}}.

  \paragraph{Subgroup generated by $X$} If $X<G$ and, e.g.,~$r$,$s$,$r^2$,$sr$
  (called \textit{words} of $X$).

  \paragraph{Theorems}
  \begin{itemize}
    \item A non-empty subset $H$ of a group $G$ is a subgroup of $G$ if and
      only if $xy^{-1}$ belongs to $H$ whenever $x$ and $y$ belong to $H$.
    \item The intersection of two subgroups of a group is itself a subgroup.
    \item Every subgroup of $\mathbb{Z}$ is cyclic. Every subgroup of a cyclic
      group is cyclic.
  \end{itemize}

  \section{Permutations}
  A \textit{permutation} is a bijection from a set $X$ to itself (e.g.,~replace
  all $3$s with $1$s).  The collecticon of \textit{all} permutations of
  $X$ forms a group $S_x$ under composition of functions (who each perform one
  specific permutation). When $X$ consists of the first $n$ positive integers,
  we get the \textbf{\textit{symmetric group}} $S_n$ of degree $n$ and order
  $n!$. $S_3$ is not abelian

  $(a_1a_2\ldots a_k)$ is called a \textbf{\textit{cyclic permutation}},
  sending $a_1$ to $a_2$, $\ldots$ , $a_k$ to $a_1$. Its length is $k$ and a
  cyclic permutation of length $k$ is called a \textbf{\textit{k-cycle}}. A
  2-cycle is called a \textbf{\textit{transposition}}. Every element of $S_n$
  can be written as many such \textbf{\textit{disjoint}}, meaning no integer is
  moved by more than one of them. Therefore they are \textit{commutative}.

  \paragraph{Theorems}
  \begin{itemize}
    \item The transpositions in $S_n$ together generate $S_n$.
    \item The transpositions $(12), (13), \ldots, ({1n)}$ together generate $S_n$.
    \item The transpositions $(12), (23), \ldots, (n-1,n)$ together generate $S_n$.
    \item The transposition $(12)$ and the $n$-cycle $(12 \ldots n)$ together
      generate $S_n$.
  \end{itemize}

  A given element of $S_n$ can be written as a product of transpositions in
  many different ways. But the number of transpositions is always even or
  always odd.

  \paragraph{Theorems}
  \begin{itemize}
    \item The even permutations in $S_n$ form a subgroup of order $n!/2$ called
      the \textbf{\textit{alternating group $A_n$}} of degree $n$.
    \item For $n\ge 3$ the 3-cycles generate $A_n$.
  \end{itemize}

  \section{Isomorphisms}

\end{document}

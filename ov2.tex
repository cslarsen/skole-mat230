\documentclass[a4paper,norsk,12pt]{article}
\usepackage[utf8]{inputenc}

% Oppsett for norsk
\usepackage[norsk]{babel}
\usepackage{times}
\usepackage[T1]{fontenc}
\usepackage{parskip}
\DeclareUnicodeCharacter{00A0}{ }
\newcommand{\strek}{\textthreequartersemdash}

% Andre pakker
\usepackage{oving}
\usepackage{amsmath}
\usepackage{amssymb}
\usepackage{varioref}
\usepackage{subcaption}
\usepackage{units}


\title{MAT230 --- Grupper og symmetri}
\subtitle{Obligatorisk innlevering 2}
\author{Christian Stigen}
\date{UiS, 18.~mars, 2016}

\begin{document}
\maketitle

\oppgave{2a}

$\phi{(n)}$ er antall $0 < x \leqslant n$ slik at $\text{gcd}{(x,n)} = 1$.
Funksjonen er multiplikativ, så $\phi{(10)} = \phi{(2)}\phi{(5)}$:
\begin{align*}
  \text{gcd}(1, 2) = 1 & \qquad\text{gcd}(1, 5) = 1 \\
                       & \qquad\text{gcd}(2, 5) = 1 \\
                       & \qquad\text{gcd}(3, 5) = 1 \\
                       & \qquad\text{gcd}(4, 5) = 1
\end{align*}
Vi får $k = \phi{(2)}\phi{(5)} = 1\cdot4 =4 $.

\oppgave{2b}
Vi vet $|R_10| = \phi{(10)} = 4$, og elementene er alle $0<x\geqslant 10$ slik
at $\text{gcd}(x,10)=1$ (trikset over kan ikke brukes for å finne selve
elementene). Dermed er $R_{10} = \{ 1,3,7,9 \}$. Vi ser at
\begin{align*}
  3^0 &= 1         & 3^2 &= 9 \\
  3^2 &= 9         & 3^3 \Mod{10} &= 7 \\
  3^4\Mod{10} &= 1 & 3^5\Mod{10} &= 3
\end{align*}
Altså ser vi at $\langle 3 \rangle = R_{10}$, altså er $R_{10}$ er generert av
3 og syklisk.

Gruppen $\mathbb{Z}_{k=4} = \{0,1,2,3\}$ under addisjon modulo fire. Vi ser at

Vi ser at 

\end{document}

\documentclass[a4paper,norsk,twocolumn,10pt]{article}
\usepackage[utf8]{inputenc}

% Oppsett for norsk
\usepackage[norsk]{babel}
\usepackage{times}
\usepackage[T1]{fontenc}
\usepackage{parskip}
\DeclareUnicodeCharacter{00A0}{ }
\newcommand{\strek}{\textthreequartersemdash}

% Andre pakker
\usepackage{amsmath}
\usepackage{amssymb}
\usepackage{varioref}
\usepackage{subcaption}
\usepackage{units}
\usepackage{todo}

% Double brackets
\usepackage{stmaryrd}
\usepackage[hidelinks]{hyperref}

% Footnote symbols
\usepackage[symbol*]{footmisc}

\newcommand{\Mod}[1]{\ (\text{mod}\ #1)}

% \mathscr
\usepackage{mathrsfs}


\title{MAT230 Grupper og symmetri\\ \textit{Obligatorisk innlevering 2}}
\author{Christian Stigen}
\date{UiS, 18.~mars, 2016}

\begin{document}
\maketitle

\section*{Oppgave 2}
\paragraph{\textbf{a)}}
$\phi{(n)}$ er antall $0 < x \leqslant n$ slik at $\text{gcd}{(x,n)} = 1$.

Funksjonen er multiplikativ; $\phi{(10)} = \phi{(2)} \phi{(5)}$ og
\begin{align*}
  \text{gcd}(1, 2) = 1 \quad & \text{gcd}(1, 5) = 1 \\
                       & \text{gcd}(2, 5) = 1 \\
                       & \text{gcd}(3, 5) = 1 \\
                       & \text{gcd}(4, 5) = 1
\end{align*}
gir $k = 1\cdot4 =4 $.

\paragraph{\textbf{b)}}
Vi vet $|R_10| = \phi{(10)} = 4$, og elementene er alle $0<x\geqslant 10$ slik
at $\text{gcd}(x,10)=1$ (trikset over kan ikke brukes for å finne selve
elementene). Dermed er $R_{10} = \{ 1,3,7,9 \}$. Vi ser at
\begin{align*}
  3^0 \Mod{10} &= 1 \\
  3^1 \Mod{10} &= 3 \\
  3^2 \Mod{10} &= 9 \\
  3^3 \Mod{10} &= 7 \\
  3^4 \Mod{10} &= 1 \\
  3^5 \Mod{10} &= 3 \\
  \ldots
\end{align*}
Altså er $\langle 3 \rangle = R_{10}$ og $R_{10}$ er syklisk. 
$(\mathbb{Z}_{k=4}, +_4) = \{0,1,2,3\}$ er generert av $\langle 1 \rangle$ og
er dermed også syklisk.

Dersom vi tar hvert element fra $\mathbb{Z}_4$ og setter inn i
$\phi{(3^m\Mod{10})}$ så får vi alle elementene i $R_{10}$. Vi har altså laget
en én-til-én korrespondanse mellom elementene \textit{den ene veien}. Da dette
er endelige sett kan vi bare definere at å gå andre veien gjøres ved et
oppslag. Dermed må $R_{10}$ være isomorf med $\mathbb{Z}_4$.

I eksempel (iv) i boka, s.~34, står det at enhver syklisk gruppe $G$ av orden
$n$ er isomorfisk med $\mathbb{Z}_n$. Hvis $\langle x \rangle = G$, så
definerer man $\phi\colon G \rightarrow \mathbb{Z}_n$ ved $\phi{(x^m)} = m
\Mod{n}$. I vårt tilfelle er $x=3$ og $n=4$.

\section*{Oppgave 3}
\paragraph{\textbf{a)}}
Det er tydelig at $\vec{a} \thicksim \vec{a}$ for enhver $a\in \mathbb{R}^3$,
fordi $a_3=a_3$.

Hvis $\vec{a} \thicksim \vec{b}$, så må også $\vec{b} \thicksim \vec{a}$, fordi
$a_3=b_3$ er det samme som $b_3=a_3$, og dette gjelder for alle $a,b \in
\mathbb{R}^3$.

Hvis $\vec{a} \thicksim \vec{b}$ og $\vec{b} \thicksim \vec{c}$, så må
$\vec{a} \thicksim \vec{c}$ fordi $a_3=b_3$ og $b_3=c_3$ impliserer at
$a_3=c_3$.

Da alle punktene over er innfridd så har vi vist at $\thicksim$ er en
ekvivalensrelasjon på $\mathbb{R}^3$.

\paragraph{\textbf{b)}}
For en vilkårlig $\vec{a} \in \mathbb{R}^3$ så vil $\mathscr{R}(\vec{a})$ være
\textit{ekvivalensklassen} til $\vec{a}$, det vil si alle vektorer $\vec{b}$
som har samme $z$-komponent $b_3=a_3$. Med andre ord er dette en
\textbf{\textit{avbildning av vektorene i $xy$-planet}}
definert ved $z=a_3$.

\section*{Oppgave 6}
For at $\thicksim$ skal være en ekvivalensrelasjon på $X$ må vi ha (1) $x\in X$
slik at $x \thicksim x$, (2) hvis $x \thicksim y$ så er $y \thicksim x$ for to
$x,y \in X$ og (3) hvis $x \thicksim y$ og $y \thicksim z$ så må $x \thicksim
z$ for tre $x,y,z \in X$.

La oss først \textbf{\textit{anta}}\footnote{Oppgaven sier ikke om det er $n
\in \mathbb{N}^+$ eller $n \in \mathbb{N}^0$. Vi velger $\mathbb{N}^+$, ellers
blir oppgaven triviell.} at det finnes en isomorfisme og et tall $n \in
\mathbb{N}^+$slik at $\phi\colon \mathbb{Z}^n \times H_n \rightarrow
\mathbb{Z}^n \times H_2$, for to vilkårlige $H<G$.  (1) Da vil $\mathbb{Z}^n
\times H \rightarrow \mathbb{Z}^n \times H$ være en bijekson på seg selv.

(2) Da $\phi$ er en bijeksjon må det finnes en invers $\phi^{-1}$, per
definisjon: $H_1 \thicksim H_2$ impliserer at det finnes $H_2 \thicksim H_1$.

(3) Hvis $\phi\colon H \rightarrow H'$ og $\psi\colon H' \rightarrow H''$ begge
er isomorfismer, så må komposisjonen $\psi\phi\colon H \rightarrow H''$ også
være en isomorfisme.

Det som nå gjenstår er å vise at isomorfismen $\phi$ faktisk finnes.

\end{document}

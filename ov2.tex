\documentclass[a4paper,norsk,12pt]{article}
\usepackage[utf8]{inputenc}

% Oppsett for norsk
\usepackage[norsk]{babel}
\usepackage{times}
\usepackage[T1]{fontenc}
\usepackage{parskip}
\DeclareUnicodeCharacter{00A0}{ }
\newcommand{\strek}{\textthreequartersemdash}

% Andre pakker
\usepackage{oving}
\usepackage{amsmath}
\usepackage{amssymb}
\usepackage{varioref}
\usepackage{subcaption}
\usepackage{units}


\title{MAT230 --- Grupper og symmetri}
\subtitle{Obligatorisk innlevering 2}
\author{Christian Stigen}
\date{UiS, 18.~mars, 2016}

\begin{document}
\maketitle

\oppgave{2a}

$\phi{(n)}$ er antall $0 < x \leqslant n$ slik at $\text{gcd}{(x,n)} = 1$.
Funksjonen er multiplikativ, så $\phi{(10)} = \phi{(2)}\phi{(5)}$:
\begin{align*}
  \text{gcd}(1, 2) = 1 & \qquad\text{gcd}(1, 5) = 1 \\
                       & \qquad\text{gcd}(2, 5) = 1 \\
                       & \qquad\text{gcd}(3, 5) = 1 \\
                       & \qquad\text{gcd}(4, 5) = 1 \\
\end{align*}
Da $\phi{(2)}=1$ og $\phi{(5)}=4$ er $k=4$.

\end{document}

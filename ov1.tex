\documentclass[a4paper,norsk,12pt]{article}
\usepackage[utf8]{inputenc}

% Oppsett for norsk
\usepackage[norsk]{babel}
\usepackage{times}
\usepackage[T1]{fontenc}
\usepackage{parskip}
\DeclareUnicodeCharacter{00A0}{ }
\newcommand{\strek}{\textthreequartersemdash}

% Andre pakker
\usepackage{oving}
\usepackage{amsmath}
\usepackage{amssymb}
\usepackage{varioref}
\usepackage{subcaption}
\usepackage{units}


\title{MAT23   --- Grupper og symmetri}
\subtitle{Obligatorisk innlevering 1}
\author{Christian Stigen}
\date{UiS, 22.~februar, 2016}

\begin{document}
\maketitle

% G={1,4,7,13}, delmengde av Z_15
\oppgave{1a}

$G = \{ 1, 4, 7, 13 \}$ er en delmengde av $\mathbb{Z}_{15}$.

For at $G$ skal være en gruppe under multiplikasjon modulo 15, må vi vise at
multiplikasjonen er \textit{assosiativ}, at det finnes et
\textit{identitetselement i $G$} og at det finnes en \textit{invers} for alle
elementer i $G$.

For tre vilkårlige element i $G$ må vi ha at $(xy)z = x(yz)$.

Eksistensen av et \textit{identitetselement} $e \in G$ som oppfyller $xe = x = ex$ for
alle $x$ i $G$. Vi ser umiddelbart at $e = 1$.

Den \textit{inverse} operasjonen krever at hvert element $x \in G$ har en
invers $x^{-1} \in G$ slik at $x^{-1}x = e = xx^{-1}$.

\oppgave{1b}
\oppgave{1c}

\oppgave{2a}
\oppgave{2b}
\oppgave{2c}

\oppgave{3a}
\oppgave{3b}
\oppgave{3c}
\oppgave{3d}

\end{document}

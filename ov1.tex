\documentclass[a4paper,norsk,12pt]{article}
\usepackage[utf8]{inputenc}

% Oppsett for norsk
\usepackage[norsk]{babel}
\usepackage{times}
\usepackage[T1]{fontenc}
\usepackage{parskip}
\DeclareUnicodeCharacter{00A0}{ }
\newcommand{\strek}{\textthreequartersemdash}

% Andre pakker
\usepackage{oving}
\usepackage{amsmath}
\usepackage{amssymb}
\usepackage{varioref}
\usepackage{subcaption}
\usepackage{units}


\title{MAT23   --- Grupper og symmetri}
\subtitle{Obligatorisk innlevering 1}
\author{Christian Stigen}
\date{UiS, 22.~februar, 2016}

\begin{document}
\maketitle

$G = \{ 1, 4, 7, 13 \}$ er en delmengde av $\mathbb{Z}_{15}$.

\oppgave{1a}

Vi begynner med multiplikasjonstabellen modulo 15.

\begin{table}[htp]
  \centering
  \begin{tabular}{r|rrrr}
       &  1 &  4 &  7 & 13 \\ \hline
     1 &  1 &  4 &  7 & 13 \\
     4 &  4 &  1 & 13 &  7 \\
     7 &  7 & 13 &  4 &  1 \\
    13 & 13 &  7 &  1 &  4 \\
  \end{tabular}
\end{table}

Først og fremst ser vi et operasjonen er \textit{lukket}, det vil si at for
alle $x,y \in G$ så er $xy \in G$. Dette ser vi av tabellen over.

\textit{Identitetselementet} må være unikt i $G$ og oppfylle $xe = x = ex$. Vi
ser umiddelbart at $e=1$.

Operasjonen må være \textit{assosiativ}, altså for tre vilkårlige element $a, b, c \in
G$ så må $(ab)c = a(bc)$. \todo{Vis dette}

Til slutt må vi vise at det finnes en \textit{invers} operasjon slik at hvert
element $x \in G$ har en invers $x^{-1} \in G$ slik at $x^{-1}x = e = xx^{-1}$.

\oppgave{1b}
\oppgave{1c}

\oppgave{2a}
\oppgave{2b}
\oppgave{2c}

\oppgave{3a}
\oppgave{3b}
\oppgave{3c}
\oppgave{3d}

\end{document}

\documentclass[a4paper,norsk,12pt]{article}
\usepackage[utf8]{inputenc}

% Oppsett for norsk
\usepackage[norsk]{babel}
\usepackage{times}
\usepackage[T1]{fontenc}
\usepackage{parskip}
\DeclareUnicodeCharacter{00A0}{ }
\newcommand{\strek}{\textthreequartersemdash}

% Andre pakker
\usepackage{oving}
\usepackage{amsmath}
\usepackage{amssymb}
\usepackage{varioref}
\usepackage{subcaption}
\usepackage{units}
\usepackage{todo}

% Double brackets
\usepackage{stmaryrd}
\usepackage[hidelinks]{hyperref}

% Footnote symbols
\usepackage[symbol*]{footmisc}


\title{MAT23   --- Grupper og symmetri}
\subtitle{Obligatorisk innlevering 1}
\author{Christian Stigen}
\date{UiS, 22.~februar, 2016}

\begin{document}
\maketitle

\oppgave{1a}
Vi ser umiddelbart at identitetselementet $e=1$ som tilfredstiller $xe=x=ex$
(bevis under) og skriver ut multiplikasjonstabellen, for å gjøre resten
enklere.

\begin{table}[htp]
  \centering
  \begin{tabular}{r|rrrr}
       &  e &  4 &  7 & 13 \\ \hline
     e &  e &  4 &  7 & 13 \\
     4 &  4 &  e & 13 &  7 \\
     7 &  7 & 13 &  4 &  e \\
    13 & 13 &  7 &  e &  4 \\
  \end{tabular}
\end{table}

$G$ er en gruppe under multiplikasjon modulo 15 fordi
\begin{itemize}
  \item vi har et unikt identitetselement $e=1 \in G$ som tilfredstiller
    $xe=x=ex$, fordi $x\cdot1 \mod{n} = 1\cdot x \mod{n} = x \mod{n}$,
  \item multiplikasjonen modulo 15 er \textit{lukket}, altså vil $a\cdot_{15} b
    \in G$ for alle $a,b \in G$ (det ser vi av tabellen),
  \item hvert element har en \textit{invers} $x^{-1}$ slik at
    $x^{-1}x=e=xx^{-1}$ fordi $1\cdot1=e$, $4\cdot4=e$, $7\cdot13=e=13\cdot7$
    (vi gir ingen generell operasjon for dette; siden gruppen vår er så liten
    er det nok å gå gjennom alle mulighetene),
  \item multiplikasjon modulo 15 er \textit{assosiativ}, altså $(xy)z = x(yz)$
    eller $(x\cdot_n y)\cdot_n z = x\cdot_n(y\cdot_n z)$ (se merknad under).
\end{itemize}

\paragraph{Assosivitet for multiplikasjon modulo n} kan bevises på flere måter.
Den \textit{minst} tilfredstillende jeg fant\footnote{Se
\url{https://proofwiki.org/wiki/Modulo_Multiplication_is_Associative}},
men som også er enklest, er å
bruke $\llbracket a \rrbracket_n \equiv a \mod {n}$. Gitt $x,y,z < n$ har vi
\begin{align*}
  \llbracket a \rrbracket_n \cdot_n \llbracket b \rrbracket_n &\equiv
    \llbracket ab \rrbracket_n \\
    \\
    \left(\llbracket x \rrbracket_n \cdot_n \llbracket y \rrbracket_n\right)
    \cdot_n \llbracket z \rrbracket_n &= 
      \llbracket xy \rrbracket_n \cdot_n \llbracket z \rrbracket_n \\
      &= \llbracket (xy)z \rrbracket_n & \text{per definisjon} \\
      &= \llbracket x(yz) \rrbracket_n & \text{assosivitet for
    heltallsmultiplikasjon} \\
      &= \llbracket x \rrbracket_n \cdot_n \llbracket yz \rrbracket_n \\
      &= \llbracket x \rrbracket_n \cdot_n \left( \llbracket y \rrbracket_n
    \cdot_n \llbracket z \rrbracket_n \right)
\end{align*}
Oppgaven ber vel ikke eksplisitt om å føre et slikt bevis, så derfor er jeg
såpass unøyaktig og stjeler et bevis fra før. Det som står i boken for addisjon
(Armstrong, s.~13) kunne også blitt brukt, men jeg synes den over er enklere.

\oppgave{1b}
\oppgave{1c}

\oppgave{2a}
\oppgave{2b}
\oppgave{2c}

\oppgave{3a}
\oppgave{3b}
\oppgave{3c}
\oppgave{3d}

\end{document}
